\addchap{Vorwort}
\label{chap:vorwort}
Schon vor meinem Studium beschäftigte ich mich mit den Möglichkeiten moderner 3D-Grafik. Ich erstellte Modifikationen für Computerspiele, wie neue, höher auflösende Texturen, durch die sich die visuelle Qualität deutlich steigern ließ. Durch das Studium lernte ich die technischen Aspekte kennen und experimentierte mit eigenen kleinen Anwendungen in OpenGL und WebGL. Daher war mir klar, dass ich in meiner Bachelorarbeit gern tiefer in dieses Thema einsteigen würde. Der Betreuer dieser Arbeit, Herr Stephan Rehfeld, brachte mich auf die Idee das Thema WebGL mit der Anbindung an ein Realtime Interactive System (RIS) zu kombinieren. Die Beuth Hochschule hat in den letzten Jahren in Kooperation mit der Universität Würzburg ein neuartiges RIS entwickelt, welches auch bereits auf dem "`Holodeck"' der Beuth Hochschule verwendet wird. Herr Rehfeld ist Teil dieses Teams, wodurch ich bei Fragen zum System schnelle und präzise Antworten bekam. Danke dafür noch einmal an ihn.

Solch ein RIS könnte komplexe Berechnungen wie die Physik übernehmen und den clientseitigen Renderer so entlasten. Auf diese Weise wäre es möglich sehr ressourcenintensive 3D-Anwendungen im Browser auch schwächerer Clients auszuführen. Dies wird in Zukunft von größerer Relevanz sein, da die derzeitige Entwicklung weg von starren Desktop-Computern und hin zu mobilen Clients, wie Notebooks, Netbooks, Smartphones und Tablets geht. Diese Endgeräte sind mittlerweile zwar sehr leistungsstark, jedoch für aufwendige 3D-Anwendungen noch immer nur begrenzt einsetzbar und es muss für jede Plattform eine eigene Clientanwendung geschrieben werden. Über nativ im Browser ausgeführtes WebGL und ein auf einem Server laufendes RIS tun sich jedoch auch hier neue Möglichkeiten für eine komplexe Echtzeitdarstellung auf.
